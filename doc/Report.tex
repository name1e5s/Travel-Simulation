\documentclass[blue]{elegantbook}
\title{模拟旅行查询系统}
\subtitle{课程实践报告}
\author{王嘉茜·于海鑫·田静悦}
\institute{北京邮电大学计算机学院}
\date{\today}

\version{1.00}
\logo{logo.png}
\cover{cover.jpg}

\begin{document}
\maketitle
\tableofcontents
\mainmatter
\hypersetup{pageanchor=true}

\chapter{概述}
\section{任务描述}
任务要求建立一个旅客管理系统,该系统中内置了一张城市图表,旅客可以通过三种交通工具(汽车、火车、飞机)在各个城市间进行转移。用户(乘客)可以在任意时间对系统提出请求,系统应根据用户的输入为其确定一条旅游线路并输出。系统能模拟出旅客在旅行时某一时间内所处的地点和状态,并以日志的形式记录下来。用户输入应包括起点、终点、必经城市以及旅行策略。旅行策略共有如下三种:
\begin{enumerate}
	\item 花费最少策略
	\item 时间最少策略
	\item 限时花费最少策略
\end{enumerate}
\section{功能需求}
\begin{itemize}
	\item 城市总数不少于10个
	\item 建立汽车、火车和飞机的时刻表(航班表)
	\begin{enumerate}
		\item 有沿途到站及票价信息
		\item 不能太简单(不能总只是1班车次相连)
	\end{enumerate}
	\item 旅客的要求包括:起点、终点、途经某些城市和旅行策略
	
	旅行策略有:
	\begin{enumerate}
		\item 最少费用策略:无时间限制,费用最少即可
		\item 最少时间策略:无费用限制,时间最少即可
		\item 限时最少费用策略:在规定的时间内所需费用最省
	\end{enumerate}
	\item 旅行模拟查询系统以时间为轴向前推移,每10秒左右向前推进1个小时(非查询状态       的请求不计时);
	\item 不考虑城市内换乘交通工具所需时间
	\item 系统时间精确到小时
	\item 建立日志文件,对旅客状态变化和键入等信息进行记录
	\item 用图形绘制地图,并在地图上反映出旅客的旅行过程
\end{itemize}

\section{模型简化}
通过一些假设,我们可以剔除实际模型中不必要的变量,保留问题的本质和研究重点,简化模型。在这次课程实践中,我们提出了如下假设:
\begin{enumerate}
	\item 截至 2017 年,我国共有约 2000 个火车站办理客运业务,同时有约 220 个民用机场,汽车站更是不计其数。将这些场地全部添加到我们的系统内是不现实的,为了简化获取数据的难度,同时使最后的图形化界面更加简洁,我们只选取了除去港澳台外的全国 31 个行政区的省会城市进行建模。
	\item 可以注意到,对于火车来说,分段购票与直接购票的价格差较小(通常不超过票价的 10\%),因此我们可以假设只有相邻省份之间的城市拥有直达列车,并以沿途火车票价之和模拟不相邻省份之间的列车价格。
	\item 汽车不适用于远程交通,且其运输市场受路况等客观因素影响较大,难以精确求出其运行时间。我们假设只有相邻省份之间的城市可以通过汽车到达,且汽车的时间表是固定的。
\end{enumerate}

\section{解决方案说明}
在这一节内,我们对该系统使用的技术以及人员分工进行描述。
\subsection{编程语言}
该系统主体部分使用 C++ 进行开发,图形部分采用 Qt 框架以实现跨平台特性。爬虫部分使用 Python 的 requests 库进行开发。
\subsection{人员分工}
\begin{itemize}
	\item \textbf{于海鑫}\ 算法实现以及爬虫部分
	\item \textbf{王嘉茜}\ 图形界面
	\item \textbf{田静悦}\ 日志模块
\end{itemize}
\subsection{编程规范}
\begin{itemize}
	\item \textbf{C++}部分遵循 \href{http://www.nscscc.org/uploads/soft/170318/1-1F31P20H9.docxhttp://google.github.io/styleguide/cppguide}{Google C++ Style Guide}, 通过使用 clang 提供的工具包保证代码风格的一致性
	\item \textbf{Python}部分遵循 \href{https://www.python.org/dev/peps/pep-0008/}{PEP-8},使用 IDE 提供的检测功能保证代码风格的规范性
\end{itemize}
\vspace*{3 ex}
以下将分章介绍实现细节。

\chapter{总体设计}

\chapter{算法实现}

\chapter{爬虫部分}
该系统的爬虫部分主要采用 Python 的 requests 库实现,辅以代理池等技术保证不被网站的反爬虫技术限制。最终在 12306 上爬取到车票信息 14655 条,在携程网上爬取机票信息 930 条。

\chapter{图形界面}

\chapter{日志模块}

\chapter{测试与分析}

\chapter{总结}

\end{document}