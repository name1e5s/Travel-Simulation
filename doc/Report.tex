\documentclass[lang=cn,blue]{elegantbook}
\title{模拟旅行查询系统}
\subtitle{课程实践报告}

\author{王嘉茜·于海鑫·田静悦}
\institute{北京邮电大学计算机学院}
\date{\today}

\version{1.00}
\logo{logo.png}
\cover{cover.jpg}

\usepackage{algorithm}
\usepackage{algpseudocode}
\usepackage{listings}
\newfontfamily\courier{Courier New}
\lstset{linewidth=1.1\textwidth,
	numbers=left,
	basicstyle=\small\courier,
	numberstyle=\tiny\courier,
	keywordstyle=\color{blue}\courier,
	commentstyle=\it\color[cmyk]{1,0,1,0}\courier, 
	stringstyle=\it\color[RGB]{128,0,0}\courier,
	frame=single,
	breaklines,
	extendedchars=false, 
	xleftmargin=2em,xrightmargin=2em, aboveskip=1em,
	tabsize=4, 
	showspaces=false
	basicstyle=\small\courier
}

\begin{document}
\maketitle
\tableofcontents
\mainmatter
\hypersetup{pageanchor=true}

\chapter{概述}
\section{任务描述}
任务要求建立一个旅客管理系统,该系统中内置了一张城市图表,旅客可以通过三种交通工具(汽车、火车、飞机)在各个城市间进行转移。用户(乘客)可以在任意时间对系统提出请求,系统应根据用户的输入为其确定一条旅游线路并输出。系统能模拟出旅客在旅行时某一时间内所处的地点和状态,并以日志的形式记录下来。用户输入应包括起点、终点、必经城市以及旅行策略。旅行策略共有如下三种:
\begin{enumerate}
	\item 花费最少策略
	\item 时间最少策略
	\item 限时花费最少策略
\end{enumerate}
\section{功能需求}
\begin{itemize}
	\item 城市总数不少于10个
	\item 建立汽车、火车和飞机的时刻表(航班表)
	\begin{enumerate}
		\item 有沿途到站及票价信息
		\item 不能太简单(不能总只是1班车次相连)
	\end{enumerate}
	\item 旅客的要求包括:起点、终点、途经某些城市和旅行策略
	
	旅行策略有:
	\begin{enumerate}
		\item 最少费用策略:无时间限制,费用最少即可
		\item 最少时间策略:无费用限制,时间最少即可
		\item 限时最少费用策略:在规定的时间内所需费用最省
	\end{enumerate}
	\item 旅行模拟查询系统以时间为轴向前推移,每10秒左右向前推进1个小时(非查询状态       的请求不计时);
	\item 不考虑城市内换乘交通工具所需时间
	\item 系统时间精确到小时
	\item 建立日志文件,对旅客状态变化和键入等信息进行记录
	\item 用图形绘制地图,并在地图上反映出旅客的旅行过程
\end{itemize}

\section{模型简化}
通过一些假设,我们可以剔除实际模型中不必要的变量,保留问题的本质和研究重点,简化模型。在这次课程实践中,我们提出了如下假设:
\begin{enumerate}
	\item 截至 2017 年,我国共有约 2000 个火车站办理客运业务,同时有约 220 个民用机场,汽车站更是不计其数。将这些场地全部添加到我们的系统内是不现实的,为了简化获取数据的难度,同时使最后的图形化界面更加简洁,我们只选取了除去港澳台外的全国 31 个行政区的省会城市进行建模。
	\item 可以注意到,对于火车来说,分段购票与直接购票的价格差较小(通常不超过票价的 10\%),因此我们可以假设只有相邻省份之间的城市拥有直达列车,并以沿途火车票价之和模拟不相邻省份之间的列车价格。
	\item 汽车不适用于远程交通,且其运输市场受路况等客观因素影响较大,难以精确求出其运行时间。我们假设只有相邻省份之间的城市可以通过汽车到达,且汽车的时间表是固定的。
\end{enumerate}

\section{解决方案说明}
在这一节内,我们对该系统使用的技术以及人员分工进行描述。
\subsection{编程语言}
该系统主体部分使用 C++ 进行开发,图形部分采用 Qt 框架以实现跨平台特性。爬虫部分使用 Python 的 requests 库进行开发。
\subsection{人员分工}
\begin{itemize}
	\item \textbf{于海鑫}\ 算法实现以及爬虫部分
	\item \textbf{王嘉茜}\ 图形界面
	\item \textbf{田静悦}\ 日志模块
\end{itemize}
\subsection{编程规范}
\begin{itemize}
	\item \textbf{C++}部分遵循 \href{http://www.nscscc.org/uploads/soft/170318/1-1F31P20H9.docxhttp://google.github.io/styleguide/cppguide}{Google C++ Style Guide}, 通过使用 clang 提供的工具包保证代码风格的一致性
	\item \textbf{Python}部分遵循 \href{https://www.python.org/dev/peps/pep-0008/}{PEP-8},使用 IDE 提供的检测功能保证代码风格的规范性
\end{itemize}
\vspace*{3 ex}
以下将分章介绍实现细节。

\chapter{总体设计}

\chapter{算法实现}

\section{价格最低策略}

在我们的假设内,当不存在必须经过的中间城市时,该问题简化为单源最短路径问题,该问题最常见的解法为 \textbf{Dijkstra} 算法,其复杂度为 $O(|V|^2)$。我们在此处选择的算法为 \textbf{Floyd} 算法,其伪代码见算法~\ref{floyd}~,以此将全部点对之间的价格最低路径以及价格预先计算出来,并等待查询,该算法的时间复杂度为 $O(|V|^3)$。

\begin{algorithm}[H]
	\caption{\label{floyd}The Floyd Algorithm}
	\begin{algorithmic}[1]
		\Procedure{Floyd}{} 
		\State Let $dist$ be a $|V| \times |V|$ matrix of minimum distances initialized to $\infty$
		\For{\textbf{each} vertex $v$}
		\State $dist[v][v]\gets 0$
		\EndFor
		\For{\textbf{each} edge $(u,v)$}
		\State $dist[u][v]\gets w(u,v)$ \Comment the weight of the edge $(u,v)$
		\EndFor
		\For{$k$ \textbf{from} 1 \textbf{to} $|V|$}
		\For{$i$ \textbf{from} 1 \textbf{to} $|V|$}
		\For{$j$ \textbf{from} 1 \textbf{to} $|V|$}
		\If{$dist[i][j] > dist[i][k] + dist[k][j]$}
		\State $dist[i][j] \gets dist[i][k] + dist[k][j]$
		\Comment also update our route here
		\EndIf
		\EndFor
		\EndFor
		\EndFor
		\EndProcedure
	\end{algorithmic}
\end{algorithm}

当存在中间城市时,我们可以把``源城市-中间城市、中间城市-目标城市''的最短路径作为``源城市-中间城市-目标城市''的最短路径。此时中间城市的排序就成为了关键问题,该问题与旅行商问题比较类似。我们使用广为人知的模拟退火算法来处理此问题,模拟退火算法的一般流程见算法~\ref{sa}~。对于该问题,\textbf{ENERGY} 函数就是随机生成的路径序列可以达到的最低价格,而我们使用的\textbf{ACCEPT}函数的定义如下:
\begin{lstlisting}[language=C++]
static inline double accept_rate(const int &new_energy, const int &old_energy, const double &temperature) {
	return old_energy > new_energy ? exp((old_energy - new_energy) / temperature) : 1.0f;
}
\end{lstlisting}
后续的时间最短策略使用的算法也是模拟退火算法,唯一修改的只有\textbf{ENERGY},恕不再进行详细说明。

\begin{algorithm}[H]
	\caption{\label{sa}The Simulated Annealing Algorithm}
\begin{algorithmic}[1]
	\Procedure{SimulatedAnnealing}{}
	\State $T\gets T_{max}$
	\State $best\gets$ \textbf{\Call{INIT}{$ $}}
	
	\While{$T>T_{min}$}
	\State $next\gets $ \textbf{\Call{NEIGHBOUR}{$T, best$}}
	\State $\Delta E\gets$ \textbf{\Call{ENERGY}{$next$}} $-$ \textbf{\Call{ENERGY}{$best$}}
	\If{$\Delta E < 0$}
	\State $best\gets next$
	\ElsIf{\Call{random}{$ $} $<$ \textbf{\Call{ACCEPT}{$T,\Delta E$}}}
	\State $best\gets next$
	\EndIf
	\State $T\gets$ \textbf{\Call{COOLING}{$T,best$}}
	\EndWhile \\
	\Return $best$
	\EndProcedure
\end{algorithmic}
\end{algorithm}


\chapter{爬虫部分}
该系统的爬虫部分主要采用 Python 的 requests 库实现,辅以代理池等技术保证不被网站的反爬虫技术限制。最终在 12306 上爬取到车票信息 14655 条,在携程网上爬取机票信息 930 条。该部分的源码现已开源,网址为 \href{https://github.com/kuso-kodo/kuso\_12306}{https://github.com/kuso-kodo/kuso\_12306}。

\section{火车票信息}
火车票的全部信息来自于铁总的 \href{https://www.12306.cn}{12306}。该部分代码位于 \href{https://github.com/kuso-kodo/kuso_12306/blob/master/src_12306/fetch.py}{fetch.py}。

12306 查询火车票的网址结构如下:
\begin{center}
	\begin{tabular}{cc}
		\hline
		\makebox[0.5\textwidth][c]{参数}	&  \makebox[0.2\textwidth][c]{含义} \\ \hline
		https://kyfw.12306.cn/otn/leftTicket/query & 查询网址 \\
		leftTicketDTO.train\_date & 要查询的日期 \\
		leftTicketDTO.from\_station & 始发站 \\
		leftTicketDTO.to\_station & 终点站 \\
		purpose\_codes & 乘客类型 \\ \hline
	\end{tabular}
\end{center}
需要注意的是站点需要使用内部电码来表示,获取内部电码的方式是访问 \href{https://kyfw.12306.cn/otn/resources/js/framework/station\_name.js}{https://kyfw.12306.cn/otn/resources/js/framework/station\_name.js} 获取形如``@bjb|北京北|VAP|beijingbei|bjb|0''的信息之后通过正则表达式获取所需信息。
访问该 API 返回的结果为 JSON,结果内包含了查询日期内全部车次的信息。使用恰当的方式提取即可。返回结果示例如下:
\begin{lstlisting}[language={}]
{"data":{"flag":"1","map":{"BJP":"北京","SHH":"上海"},"result":["X2r16oP8IGH94iXTfaiBVftWDflhurjVyMtLApUAqGtb%UMWvDzNbAPvs2JJgtpCsG2y5VUJ2l5mC%0AbxIfI7x4izoLOx2%2FQrCgmSw0FTE0HJRw34xqltxHR6QvsJs8ZzcVyUtiZ6O57m%2Btt%2BS7QUfVl23M%0AYkqXZBJdFwX3qw33jZUkU0PLe%2FYfzcinUaZk6VpcbaHLuYoflN4kdE6Vk%2BPrYGpwqehRkGoTElzS%0A8naAdzOg1VyR8vffGyFCZ%2B4F9ZbIMrm0Hm1cVV%2BA2RsOgIps8IwLambwpZ76GgZZRZj0O6%2B0TkGY%0AG8DGrhYOPKg%3D|预订|240000D70500|D705|BJP|SHH|BJP|SHH|21:21|09:20|11:59|Y|R3ArWiqJWSrjfblWAofOiX3%2Bo5Q7Ud8vVMTg1pnCJxohcTRTmikvHJ7nmlo%3D|20190530|3|P4|01|04|0|0||||有|||有||有||有||||O0J0O0I0|OJOI|0|0|null"]},"httpstatus":200,"messages":"","status":true}
\end{lstlisting}

查询车票价格的 API 如下:
\begin{center}
	\begin{tabular}{cc}
		\hline
		\makebox[0.5\textwidth][c]{参数}	&  \makebox[0.2\textwidth][c]{含义} \\ \hline
		https://kyfw.12306.cn/otn/leftTicket/queryTicketPrice & 查询网址 \\
		train\_no & 火车编号 \\
		from\_station\_no & 起始站编号 \\
		to\_station\_no & 结束站 \\
		seat\_types & 火车类型 \\
		train\_date & 出行时间 \\ \hline
	\end{tabular}
\end{center}
需要注意的是,在访问该网站查询价格之前,需要以相同的参数访问 \href{	https://kyfw.12306.cn/otn/leftTicket/queryTicketPriceFL}{	https://kyfw.12306.cn/otn/leftTicket/queryTicketPriceFL} 一次才可以获得正确的结果,否则只能收到出错的信息。这里的大部分参数都可以在上一个 API 的回应中找到。该 API 的返回结果也是 JSON,示例如下:
\begin{lstlisting}[language={}]
{"data":{"9":"17480","A9":"¥1748.0","M":"¥933.0","O":"¥553.0","OT":[],"WZ":"¥553.0","train_no":"240000G1010K"},"httpstatus":200,"messages":"","status":true}
\end{lstlisting}
使用恰当的方式从中提取价格即可。

通过使用 Python 对以上过程编程,最终获取到了 31 个目标城市之间的直达列车信息共 14655 条。城市列表如下:
\begin{lstlisting}[language=Python]
	city_list = ['北京', '天津', '上海', '重庆', '石家庄',
	'太原', '呼和浩特', '郑州', '长沙', '武汉',
	'哈尔滨', '长春', '沈阳', '成都', '昆明',
	'贵阳', '拉萨', '乌鲁木齐', '西安', '兰州',
	'银川', '西宁', '广州', '南宁', '海口',
	'南京', '杭州', '福州', '济南', '南昌',
	'合肥']
\end{lstlisting}

\section{机票信息}
机票的信息来自于\href{https://www.ctrip.com/}{携程网}。该网站的爬取较为简单,故不再说明,只是在此记录下核心代码:
\begin{lstlisting}[language=Python]
def get_flight_prices(site_info, time_info):
	url = 'https://flights.ctrip.com/itinerary/api/12808/products'
	headers = {'referer': 'https://flights.ctrip.com/itinerary/oneway',
	'content-type': 'application/json'}
	data = {
	"flightWay": "Oneway",
	"classType": "ALL",
	"hasChild": False,
	"hasBaby": False,
	"searchIndex": 1,
	"airportParams": [{
	"dcity": site_info['depart']['city'],
	"acity": site_info['arrive']['city'],
	"dcityname": site_info['depart']['cityname'],
	"acityname": site_info['arrive']['cityname'],
	"date": time_info['date'],
	"dcityid": site_info['depart']['cityid'],
	"acityid": site_info['arrive']['cityid']
	}]
	}
	res = None
	while not res:
		res = get_resp(url, json.dumps(data), headers)
	if 200 <= res.status_code < 300:
		return parse(res.text)
	else:
		print('CTrip Sucks')
		return []
\end{lstlisting}

为了加快爬取速度以及防止被限制,我们在这里使用公开的 IP 代理信息搭建了代理池,最终在十分钟内爬取了 31 个城市间的飞机信息共 930 条。
\chapter{图形界面}

\chapter{日志模块}

\chapter{测试与分析}

\chapter{总结}

\end{document}